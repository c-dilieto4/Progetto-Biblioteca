\documentclass[a4paper,12pt]{article}
\usepackage[utf8]{inputenc}
\usepackage[T1]{fontenc}
\usepackage[italian]{babel}
\usepackage{geometry}
\usepackage{titlesec}
\usepackage{enumitem}
\usepackage{float}
\usepackage{graphicx}
\usepackage{hyperref}
\hypersetup{
    colorlinks=true,
    linkcolor=black,    % Colore dei link interni (es. Figure, Sezioni)
    urlcolor=blue,     % Colore dei link web
    pdftitle={Documentazione SGBU}, % Titolo nei metadati del PDF
}

% Impostazioni margini
\geometry{
    top=2.5cm,
    bottom=2.5cm,
    left=2.5cm,
    right=2.5cm
}



\begin{document}
\begin{titlepage}
    \centering
    \vspace*{1cm} 
    
    {\Large \textsc{Università degli Studi di Salerno} \par}
    
    \vspace{2.5cm} 
    
    {\huge \bfseries Sistema di Gestione Biblioteca Universitaria \par}	
    
    \vspace{2.5cm} % Spazio verticale tra titolo e sottotitolo

    \begin{center}
    {\Large Documentazione Architetturale del Sistema}
    \end{center}
    
    
    
    \vspace{2cm} % Spazio tra Titolo e Autori
    
    % --- Blocco Autori (Spostato qui e ingrandito) ---
    {\Large % Inizio ingrandimento testo
    \textbf{Autori:}\\
    Allocco Lorenzo\\
    Atripaldi Alessandro\\
    Di Lieto Christian Pio\\
    Graziosi Gerardo
    \par} % Fine ingrandimento
    % ------------------------------------------------
    
    \vspace{2cm} % Spazio tra Autori e Versione
    
    {\Large Versione 1.4 - 14 Dicembre 2025\par}
    
    \vspace*{\fill} % Spinge tutto verso l'alto bilanciando lo spazio finale

	\end{titlepage}

\tableofcontents
\newpage
\section{Commento al Diagramma delle Classi}

Il diagramma delle classi proposto rappresenta il modello statico del sistema SGBU, progettato seguendo un approccio Object-Oriented e adottando il pattern architetturale MVC (Model-View-Controller) a strati.\\

\noindent \textit{Per esaminare il Diagramma delle Classi,}\href{https://github.com/c-dilieto4/Progetto-Biblioteca/blob/main/docs/Documentazione%20Design/Diagramma%20delle%20Classi.pdf}{\textit{ clicca qui per andare alla figura}.}


\subsection{Organizzazione Logica}
Il sistema è decomposto in quattro livelli logici distinti, ciascuno con una responsabilità chiara e ben definita:

\begin{itemize}
    \item \textbf{Livello di Presentazione (View):}\\ 
    Rappresentato dalla classe \texttt{GUIView} e \texttt{MessaggiInterfaccia}. Questo livello si occupa esclusivamente dell'interazione con l'utente (Bibliotecario) e della visualizzazione dei dati, senza contenere alcuna logica di business.
    
    \item \textbf{Livello di Controllo (Controller):}\\
    La classe \texttt{GUIController} funge da orchestratore centrale. Essa non implementa la logica di dettaglio, ma coordina il flusso delle operazioni, delega le validazioni e smista le richieste ai servizi appropriati.
    
    \item \textbf{Livello di Servizio (Model/Service):}\\
    Le classi \texttt{Catalogo}, \texttt{Anagrafica} e \texttt{RegistroPrestiti} costituiscono il cuore applicativo. Esse incapsulano la logica di business, gestiscono le collezioni di dati e garantiscono l'integrità delle transazioni (es. verifica disponibilità e limiti).
    
    \item \textbf{Livello di Dominio (Domain) e Infrastruttura:}\\
    Le entità \texttt{Libro}, \texttt{Utente}, \texttt{Prestito} e \texttt{Credenziali} rappresentano i dati puri. Le interfacce \texttt{IArchivioDati}, \texttt{ILogger} e \texttt{IAutenticatore} (con le relative implementazioni concrete) isolano il sistema dai dettagli tecnici di persistenza e sicurezza.
\end{itemize}
\newpage
\subsection{Analisi di Coesione e Accoppiamento}
Le decisioni di decomposizione modulare sono state guidate dalla necessità di massimizzare la coesione interna delle classi e minimizzare l'accoppiamento tra i moduli.

\subsubsection{Coesione Funzionale}
Tutte le classi identificate presentano un livello di Coesione Funzionale (il livello più alto e desiderabile), in quanto ogni modulo è focalizzato su un singolo compito ben definito.
\begin{itemize}
    \item La classe \texttt{ValidatoreDati} si occupa esclusivamente della verifica sintattica.
