\documentclass[a4paper,12pt]{article}
\usepackage[utf8]{inputenc}
\usepackage[T1]{fontenc}
\usepackage[italian]{babel}
\usepackage{geometry}
\usepackage{titlesec}
\usepackage{enumitem}
\usepackage{float}
\usepackage{graphicx}
\usepackage{hyperref}
\hypersetup{
    colorlinks=true,
    linkcolor=black,    % Colore dei link interni (es. Figure, Sezioni)
    urlcolor=blue,     % Colore dei link web
    pdftitle={Documentazione SGBU}, % Titolo nei metadati del PDF
}

% Impostazioni margini
\geometry{
    top=2.5cm,
    bottom=2.5cm,
    left=2.5cm,
    right=2.5cm
}



\begin{document}
\begin{titlepage}
    \centering
    \vspace*{1cm} 
    
    {\Large \textsc{Università degli Studi di Salerno} \par}
    
    \vspace{2.5cm} 
    
    {\huge \bfseries Sistema di Gestione Biblioteca Universitaria \par}	
    
    \vspace{2.5cm} % Spazio verticale tra titolo e sottotitolo

    \begin{center}
    {\Large Documentazione Architetturale del Sistema}
    \end{center}
    
    
    
    \vspace{2cm} % Spazio tra Titolo e Autori
    
    % --- Blocco Autori (Spostato qui e ingrandito) ---
    {\Large % Inizio ingrandimento testo
    \textbf{Autori:}\\
    Allocco Lorenzo\\
    Atripaldi Alessandro\\
    Di Lieto Christian Pio\\
    Graziosi Gerardo
    \par} % Fine ingrandimento
    % ------------------------------------------------
    
    \vspace{2cm} % Spazio tra Autori e Versione
    
    {\Large Versione 1.4 - 14 Dicembre 2025\par}
    
    \vspace*{\fill} % Spinge tutto verso l'alto bilanciando lo spazio finale

	\end{titlepage}

\tableofcontents
\newpage
\section{Commento al Diagramma delle Classi}

Il diagramma delle classi proposto rappresenta il modello statico del sistema SGBU, progettato seguendo un approccio Object-Oriented e adottando il pattern architetturale MVC (Model-View-Controller) a strati.\\

\noindent \textit{Per esaminare il Diagramma delle Classi,}\href{https://github.com/c-dilieto4/Progetto-Biblioteca/blob/main/docs/Documentazione%20Design/Diagramma%20delle%20Classi.pdf}{\textit{ clicca qui per andare alla figura}.}


\subsection{Organizzazione Logica}
Il sistema è decomposto in quattro livelli logici distinti, ciascuno con una responsabilità chiara e ben definita:

\begin{itemize}
    \item \textbf{Livello di Presentazione (View):}\\ 
    Rappresentato dalla classe \texttt{GUIView} e \texttt{MessaggiInterfaccia}. Questo livello si occupa esclusivamente dell'interazione con l'utente (Bibliotecario) e della visualizzazione dei dati, senza contenere alcuna logica di business.
    
    \item \textbf{Livello di Controllo (Controller):}\\
    La classe \texttt{GUIController} funge da orchestratore centrale. Essa non implementa la logica di dettaglio, ma coordina il flusso delle operazioni, delega le validazioni e smista le richieste ai servizi appropriati.
    
    \item \textbf{Livello di Servizio (Model/Service):}\\
    Le classi \texttt{Catalogo}, \texttt{Anagrafica} e \texttt{RegistroPrestiti} costituiscono il cuore applicativo. Esse incapsulano la logica di business, gestiscono le collezioni di dati e garantiscono l'integrità delle transazioni (es. verifica disponibilità e limiti).
    
    \item \textbf{Livello di Dominio (Domain) e Infrastruttura:}\\
    Le entità \texttt{Libro}, \texttt{Utente}, \texttt{Prestito} e \texttt{Credenziali} rappresentano i dati puri. Le interfacce \texttt{IArchivioDati}, \texttt{ILogger} e \texttt{IAutenticatore} (con le relative implementazioni concrete) isolano il sistema dai dettagli tecnici di persistenza e sicurezza.
\end{itemize}
\newpage
\subsection{Analisi di Coesione e Accoppiamento}
Le decisioni di decomposizione modulare sono state guidate dalla necessità di massimizzare la coesione interna delle classi e minimizzare l'accoppiamento tra i moduli.

\subsubsection{Coesione Funzionale}
Tutte le classi identificate presentano un livello di Coesione Funzionale (il livello più alto e desiderabile), in quanto ogni modulo è focalizzato su un singolo compito ben definito.
\begin{itemize}
    \item La classe \texttt{ValidatoreDati} si occupa esclusivamente della verifica sintattica.
    \item La classe \texttt{AuditTrail} si occupa esclusivamente di registrare le operazioni svolte, separando il tracciamento dalla logica specifica dei prestiti.
    \item Il \texttt{RegistroPrestiti} accentra tutte le regole relative alle transazioni.
\end{itemize}

\subsubsection{Basso Accoppiamento e Gestione delle Dipendenze}
Abbiamo strutturato le relazioni per evitare un accoppiamento stretto (come il Content o Control Coupling).
\begin{itemize}
    \item Le dipendenze sono gestite tramite \textbf{Accoppiamento per Dati}, passando solo le informazioni strettamente necessarie.
    \item L'introduzione delle interfacce (\texttt{IArchivioDati}, \texttt{ILogger}, \texttt{IAutenticatore}) ha disaccoppiato il \texttt{GUIController} dalle implementazioni concrete.
\end{itemize}

\subsection{Principi di Progettazione Adottati}
Il diagramma riflette l'applicazione dei principi di buona progettazione orientata agli oggetti:

\begin{itemize}
    \item\textbf{Separazione delle Preoccupazioni (SoC):} Il sistema rispetta la separazione tra logica di presentazione, logica di business e gestione dati. La \texttt{GUIView} non contiene logica applicativa; il \texttt{GUIController} è un puro orchestratore.
    
    \item\textbf{Single Responsibility Principle (SRP):} Ogni classe ha una sola responsabilità. La divisione tra \texttt{FileArchivio} (persistenza) e \texttt{Catalogo} (logica di dominio) ne è un esempio.
    
    \item\textbf{Dependency Inversion Principle (DIP):} I moduli di alto livello dipendono dalle astrazioni. Il \texttt{GUIController} dipende da \texttt{IArchivioDati}, non dai dettagli di basso livello come \texttt{FileArchivio}, rendendo il sistema pronto per future strategie di memorizzazione (es. Database).
    
    \item \textbf{Open-Closed Principle (OCP):} Grazie alle interfacce, il sistema è "aperto all'estensione ma chiuso alla modifica". È possibile introdurre nuovi meccanismi di log o autenticazione senza alterare le classi esistenti.
\end{itemize}

\section{Commento al Diagramma di Sequenza: Registrazione Prestito}

Il diagramma dettaglia il flusso di interazione per assegnare un libro a un utente, rispettando il pattern MVC.\\

\noindent \textit{Per esaminare il Diagramma di Sequenza per la Registrazione Prestito,}\href{https://github.com/c-dilieto4/Progetto-Biblioteca/blob/main/docs/Documentazione%20Design/Diagramma%20di%20sequenza%20Registrazione%20Prestito.svg}{\textit{ clicca qui per andare alla figura}.}

\subsection{Analisi e Scelte di Progettazione}

\begin{itemize}
    \item \textbf{Gestione dell'Input e Validazione (Controller):} L'attore \textit{Bibliotecario} interagisce con la \texttt{GUIView}, che delega al \texttt{GUIController}. Quest'ultimo esegue una validazione formale tramite \texttt{ValidatoreDati} prima di coinvolgere la logica di business.
    
    \item \textbf{Logica di Business e Gestione dei Vincoli (Service Layer):} La classe \texttt{RegistroPrestiti} gestisce l'elaborazione. Un \textit{Frame di Interazione alt} modella i vincoli:
    \begin{itemize}
        \item \textbf{Limite Prestiti [FC-2]:} Verifica se l'utente ha superato il limite di prestiti.
        \item \textbf{Disponibilità [FC-1]:} Verifica la presenza di copie fisiche.
    \end{itemize}
    Se una guardia fallisce, il flusso termina con un errore.
    
    \item \textbf{Creazione e Aggiornamento di Stato:} Nel ramo di successo, il \texttt{RegistroPrestiti} istanzia il nuovo \texttt{Prestito} e aggiorna \texttt{Libro} e \texttt{Utente} invocando i loro metodi, mantenendo l'incapsulamento.
    
    \item \textbf{Tracciamento (Audit Trail):} Al termine, il controller invoca l'interfaccia \texttt{ILogger}, soddisfacendo il requisito [IF-11] senza inquinare la logica del prestito.
\end{itemize}

\newpage
\section{Commento al Diagramma di Sequenza: Registrazione Restituzione}

Il diagramma illustra il processo di chiusura di un prestito e il ripristino della disponibilità del libro.\\

\noindent \textit{Per esaminare il Diagramma di Sequenza per la Registrazione Restituzione,}\href{https://github.com/c-dilieto4/Progetto-Biblioteca/blob/main/docs/Documentazione%20Design/Diagramma%20di%20sequenza%20Restituzione%20Prestito.svg}{\textit{ clicca qui per andare alla figura}.}



\subsection{Analisi e Scelte di Progettazione}

\begin{itemize}
    \item \textbf{Orchestrazione e Delega:} Il \texttt{GUIController} delega immediatamente l'intera logica transazionale al \texttt{RegistroPrestiti}, centralizzando la responsabilità del ciclo di vita del prestito.
    
    \item \textbf{Gestione delle Eccezioni di Business:} Un \textit{Frame di Interazione opt} modella il requisito [UC-8] sui ritardi. Se la data di restituzione effettiva è successiva a quella prevista, viene eseguita la logica di gestione del ritardo.
    
    \item \textbf{Aggiornamento Consistente dello Stato:} Il \texttt{RegistroPrestiti} coordina una sequenza atomica di aggiornamenti:
    \begin{enumerate}
        \item Incremento delle copie disponibili del \texttt{Libro}.
        \item Rimozione del prestito dalla lista dell'\texttt{Utente}.
        \item Chiusura definitiva dell'oggetto \texttt{Prestito}.
    \end{enumerate}
    
    \item \textbf{Tracciamento:} Viene invocata l'interfaccia \texttt{ILogger} per storicizzare la modifica allo stato persistente.
\end{itemize}

\newpage
\section{Commento al Diagramma dei Package}

Il Diagramma dei Package illustra l'organizzazione logica del codice sorgente, strutturato per gestire la complessità e garantire la \textit{Separazione delle Preoccupazioni}.\\\\
\noindent \textit{Per esaminare il Diagramma dei Package,}\href{https://github.com/c-dilieto4/Progetto-Biblioteca/blob/main/docs/Documentazione%20Design/Diagramma%20dei%20package.pdf}{\textit{ clicca qui per andare alla figura}.}


\subsection{Organizzazione Architetturale}
Il sistema segue un'architettura a strati che riflette il pattern MVC:

\begin{itemize}
    \item \textbf{it.unisa.sgbu.domain (Dominio):} Il nucleo stabile del sistema. Contiene le entità (\texttt{Libro}, \texttt{Utente}, \texttt{Prestito}, \texttt{Credenziali}) e utility di validazione. È indipendente da tutti gli altri package.
    
    \item \textbf{it.unisa.sgbu.service (Logica di Business):} Contiene i gestori (\texttt{RegistroPrestiti}, \texttt{Catalogo}, \texttt{Anagrafica}). Dipende dal dominio ma non conosce l'interfaccia utente.
    
    \item \textbf{it.unisa.sgbu.io (Infrastruttura e Servizi):} Raggruppa componenti tecnici per persistenza, logging e sicurezza (\texttt{IArchivioDati}, \texttt{FileArchivio}, ecc.). L'isolamento permette di modificare le tecnologie senza impatto sul sistema.
    
    \item \textbf{it.unisa.sgbu.gui (Presentazione e Controllo):} Contiene componenti tecnici per la presentazione e il controllo dell'interfaccia (\texttt{GUIView}, \texttt{GUIController}, ecc). Agisce da livello superiore orchestrando le interazioni.
    
    \item \textbf{it.unisa.sgbu.app (Avvio):} Contiene solo il \texttt{Main} per il bootstrap.
\end{itemize}
\end{document}
